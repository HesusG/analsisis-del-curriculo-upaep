\documentclass[man, floatsintext]{apa7}
\usepackage[spanish]{babel}
\usepackage[utf8]{inputenc}
\usepackage[T1]{fontenc}
\usepackage{url}
\usepackage{csquotes}
\usepackage{graphicx}
\usepackage{setspace}
\usepackage{apacite}
\usepackage{booktabs}
\usepackage[hyphens]{url}
\usepackage[breaklinks=true]{hyperref}
\usepackage{fancyhdr}
\usepackage{epstopdf}
\usepackage{setspace}
\usepackage[nomarkers]{endfloat}
\usepackage{fontspec}
\setmainfont{Arial}
\usepackage{ragged2e}
\usepackage{microtype}
\tolerance=1000
\emergencystretch=\maxdimen
\hyphenpenalty=1000
\exhyphenpenalty=100
\geometry{margin=2cm}
\setstretch{1.3}

\renewcommand{\efloatseparator}{\mbox{}}

% =======================
% Datos del trabajo
\newcommand{\myTitle}{Proyecto Final: Análisis Crítico del Currículum}
\newcommand{\myAuthor}{Hesus Garcia Cobos}
\newcommand{\myAffiliation}{Universidad Popular Autónoma de Puebla}
\newcommand{\myCourse}{147-139-PED605NE-702: Análisis del Currículum}
\newcommand{\myProfessor}{Dra. Melissa Isaaly Mendoza Bernabe}
\newcommand{\myDate}{20 de septiembre de 2025}

\fancypagestyle{fancyheader}{
  \fancyhf{}
  \fancyhead[R]{\thepage}
  \renewcommand{\headrulewidth}{0pt}
}

% =======================
\begin{document}

% Página de título estilo APA
\thispagestyle{empty}
\vspace*{2in}
\begin{center}
    \textbf{\Large \myTitle} \\[24pt]
    \myAuthor \\[12pt]
    \myAffiliation \\[12pt]
    \myCourse \\[12pt]
    \myProfessor \\[12pt]
    \myDate
\end{center}

\newpage
\pagestyle{fancy}
\fancyhf{}
\fancyhead[R]{\thepage}
\doublespacing

% =======================
\justifying
\begin{justify}
\section{Introducción}


El currículum es  ``el contenido cultural que las instituciones educativas tratan de difundir en quienes las frecuentan, así como los efectos que dicho contenido provoque en sus receptores.'' (Sacristán, 2010). No es neutro ni estático. Refleja decisiones sobre qué conocimiento es valioso y con qué fines. Mi perspectiva integra tres elementos clave: la teoría de multiliteracidades de Cope y Kalantzis, una visión tecnológico-educativa y un enfoque socio-crítico. Esta mirada reconoce que el currículum debe responder a la diversidad de modos de comunicación actuales: lingüísticos, visuales, digitales y auditivos. Los estudiantes no son receptores pasivos, sino diseñadores activos de significado. 


\begin{itemize}
    \item \textbf{Tipo de currículum}  
    \begin{itemize}
        \item Currículum operacional que trasciende lo oficial para convertirse en práctica transformadora.
    \end{itemize}

    \item \textbf{Dimensiones principales}  
    \begin{itemize}
        \item \textbf{Social}: responde a una sociedad multicultural y digital.
        \item \textbf{Pedagógica}: integra tecnologías como herramientas de creación.
        \item \textbf{Epistemológica}: reconoce el conocimiento multimodal y colaborativo.
    \end{itemize}

    \item \textbf{Fuentes que nutren la propuesta}  
    \begin{itemize}
        \item \textbf{Sociocultural}: demandas de alfabetización múltiple.
        \item \textbf{Psicológica}: procesos de aprendizaje mediados por tecnología.
        \item \textbf{Pedagógica}: prácticas que integran creatividad y herramientas digitales.
    \end{itemize}

    \item \textbf{Enfoque socio-crítico}  
    \begin{itemize}
        \item Cuestiona constantemente: ``¿para qué enseñamos esto y a quién sirve?''.
        \item Como señala Apple (1979): el currículum nunca es neutral, representa relaciones de poder e ideología.
    \end{itemize}
\end{itemize}

\section*{Diagnóstico Curricular}

Los estudiantes de Marketing y Análisis del Consumidor en este contexto presentan características particulares:

\begin{itemize}
    \item \textbf{Actitudes}: Buscan aplicación práctica inmediata; desean que lo aprendido tenga uso real en el mercado laboral.
    
    \item \textbf{Intereses}: Se inspiran en vacantes de trabajo actuales o en negocios de familiares; desean integrarse al entorno laboral que ven proyectado.
    
    \item \textbf{Saberes previos}: Poseen capacidades de análisis y síntesis, interpretan gráficos, usan plataformas digitales como bibliotecas y bases de datos, y tienen habilidades de investigación desarrolladas.
    
    \item \textbf{Desafío pedagógico}: Aunque requieren enfoque práctico, necesitan profundidad teórica reflexiva, no solo habilidades de retorno rápido. La tensión reside en conectar ambas dimensiones.
\end{itemize}

\subsection*{Elementos específicos del currículum a abordar}

Este diseño curricular se enfoca en transformar tres dimensiones:

\begin{itemize}
    \item \textbf{Dimensión pedagógica}: De módulos independientes a secuencias integradas donde cada fase construye sobre la anterior, vinculando diagnóstico, análisis, y desarrollo de estrategia fundamentada en datos reales.
    
    \item \textbf{Dimensión epistemológica}: De conocimiento fragmentado y transmitido a conocimiento multimodal, colaborativo, y construido a partir de investigación primaria y secundaria auténtica.
    
    \item \textbf{Dimensión social}: De ejercicios controlados a participación en ecosistemas reales de investigación de mercado, donde estudiantes trabajan con pequeñas empresas o emprendimientos reales, no casos ficticios.
\end{itemize}


\subsection*{Diagnóstico inicial}
El currículo actual presenta las siguientes características:
\begin{itemize}
    \item Estructura dividida en módulos teóricos independientes
    \item Ejercicios y casos controlados por el profesor
    \item Poca conexión con datos y problemas reales del mercado
    \item Evaluación centrada en exámenes tradicionales
    \item Limitada integración de herramientas digitales y plataformas de análisis de datos
\end{itemize}

Esta estructura me recuerda a lo que Cope y Kalantzis (2012) denominan ``aprendizaje antiguo'', donde el conocimiento se presenta de manera fragmentada y, desde mi perspectiva, descontextualizada de las prácticas profesionales que veo en el mercado actual. Además, leyendo a Michael Apple en \textit{Ideología y currículo}, me pregunto qué tipo de conocimiento se está legitimando con esta estructura tradicional y a quiénes beneficia realmente mantener esta separación entre teoría académica y práctica del mercado.

Mi análisis buscará explorar cómo podría transformarse este currículo tradicional hacia uno más conectado y auténtico. Aunque aún no tengo definidos todos los elementos específicos de la propuesta, reconozco la necesidad de investigar más teorías y enfoques que permitan esta transición. Me interesa particularmente explorar:

El conectivismo de Siemens (2005) me parece un marco interesante para entender cómo el aprendizaje puede ocurrir a través de redes digitales y comunidades de práctica. Sin embargo, necesito profundizar más en cómo traducir estos principios teóricos a actividades concretas dentro del currículo. También me cuestiono, siguiendo a Apple (1979) si la integración de estas tecnologías podría reproducir nuevas formas de desigualdad o tener implicaciones de ideología.

Este proceso de análisis y propuesta se desarrollará a lo largo del semestre, lo que me permitirá reflexionar más profundamente sobre las tensiones entre la estructura curricular tradicional y las demandas de un mundo profesional cada vez más conectado y basado en datos, sin perder de vista las implicaciones ideológicas de estas transformaciones curriculares.


\subsection*{Objetivo}


Transformar el currículum de Marketing y Análisis del Consumidor de una estructura tradicional, fragmentada y centrada en la transmisión de conocimiento teórico, hacia un modelo integrado, auténtico y basado en problemas reales del mercado laboral, donde los estudiantes se conviertan en diseñadores activos de significado y actores dentro de ecosistemas de conocimiento profesionales.

% ============================================
\section{Secuencia Didáctica}
% ============================================
Esta propuesta presenta el diseño de la secuencia didáctica para el \textbf{Módulo 1: Diagnóstico Situacional} del curso de Marketing y Análisis del Consumidor. Este módulo constituye la base fundamental de la experiencia de aprendizaje integrada, donde los estudiantes desarrollan capacidades de análisis contextual crítico e identifican dinámicas estratégicas de la industria a través de investigación secundaria auténtica.

El Módulo 1 se estructura como la apertura del proceso educativo, estableciendo tanto habilidades técnicas como disposiciones reflexivas que continuarán desarrollándose en módulos posteriores. Aunque la propuesta completa comprende cuatro módulos secuenciados, en esta entrega presentamos el desarrollo detallado del Módulo 1, el cual constituye el fundamento pedagógico y conceptual para toda la secuencia curricular.

% ============================================
\section{Objetivo General del Módulo 1}
% ============================================

Que los estudiantes diagnostiquen la situación actual de una industria o categoría de consumo específica, identificando dinámicas competitivas, tendencias relevantes y oportunidades estratégicas mediante análisis integrado de información secundaria, adoptando una postura crítica sobre qué datos se priorizan y cuáles se invisibilizan en el análisis empresarial.

% ============================================
\section{Adecuación Pedagógica del Módulo 1}
% ============================================

El Módulo 1 responde a la brecha curricular identificada: la fragmentación entre teoría académica y práctica profesional. Mediante un enfoque basado en problemas auténticos (empresas reales o simuladas que requieren diagnóstico estratégico), los estudiantes aplican herramientas de inteligencia competitiva no como ejercicios escolares, sino como prácticas profesionales genuinas.

La secuencia integra:

\begin{itemize}
    \item \textbf{Conectivismo}: Los estudiantes acceden a múltiples fuentes de información (reportes sectoriales, bases de datos INEGI, análisis competitivo en línea) y construyen conocimiento colaborativamente.
    
    \item \textbf{Pensamiento crítico}: Más allá de recopilar datos, cuestionan qué revelan y qué ocultan los análisis de mercado convencionales.
    
    \item \textbf{Autenticidad}: El reto que abordan es real o realista, permitiendo que el aprendizaje tenga significado transferible.
\end{itemize}

% ============================================
\section{Análisis de Tareas}
% ============================================

\subsection*{Tabla 1: Objetivo Terminal del Módulo 1}

\begin{table}[H]
\centering
\small
\begin{tabular}{|p{2.8cm}|p{13.2cm}|}
\hline
\textbf{Meta de Aprendizaje} & 
Estudiantes diagnosticarán la situación actual de una industria/categoría de consumo, identificando dinámicas competitivas, tendencias relevantes y oportunidades estratégicas mediante análisis de información secundaria. \\
\hline

\textbf{Objetivo Terminal de Aprendizaje} & 
Estudiantes elaborarán un reporte de diagnóstico situacional que integre análisis FODA, dinámicas competitivas, tendencias de industria e identificación de oportunidades estratégicas para una empresa/categoría específica, demostrando análisis crítico sobre datos priorizados versus invisibilizados. \\
\hline
\end{tabular}
\end{table}

% ============================================
\subsection*{Tabla 2: Pasos de Procesamiento de Información (PPI) y Requisitos Previos (RP)}
% ============================================

\begin{table}[H]
\centering
\small
\begin{tabular}{|p{2.8cm}|p{6.2cm}|p{6.5cm}|}
\hline
\textbf{Paso/Requisito} & \textbf{Descripción} & \textbf{Objetivo de Aprendizaje Habilitador} \\
\hline

\textbf{RP 1} & 
Identificar la industria/categoría específica a analizar & 
Después de esta lección, los estudiantes pueden reconocer los límites y características definitorias de una industria o categoría de consumo con 85\% de precisión. \\
\hline

\textbf{RP 2} & 
Reconocer fuentes de información secundaria disponibles (bases de datos, reportes, análisis competitivo) & 
Después de esta lección, los estudiantes pueden identificar al menos 5 fuentes confiables de inteligencia de mercado específicas a su industria. \\
\hline

\textbf{PPI 1} & 
Recopilar información sobre tendencias de industria y contexto macroeconómico & 
Después de esta lección, los estudiantes pueden extraer datos relevantes de reportes sectoriales con mínimo 80\% de precisión en la selección de información pertinente. \\
\hline

\textbf{PPI 2} & 
Analizar competencia directa e indirecta & 
Después de esta lección, los estudiantes pueden mapear competidores, identificar sus propuestas de valor y diferenciar competencia directa de indirecta. \\
\hline

\textbf{PPI 3} & 
Aplicar análisis FODA contextualizado & 
Después de esta lección, los estudiantes pueden construir una matriz FODA que no sea genérica, sino que integre hallazgos específicos sobre dinámicas de la industria. \\
\hline

\textbf{PPI 4} & 
Identificar oportunidades estratégicas emergentes & 
Después de esta lección, los estudiantes pueden reconocer brechas, tendencias no explotadas y necesidades del mercado que representen oportunidades viables. \\
\hline

\textbf{PPI 5} & 
Evaluar críticamente qué información se prioriza versus se invisibiliza & 
Después de esta lección, los estudiantes pueden reflexionar críticamente sobre los sesgos en la recopilación de datos y sus implicaciones estratégicas. \\
\hline

\end{tabular}
\end{table}

% ============================================
\section{Secuencia Didáctica Operativa}
% ============================================

\subsection*{Tabla 3: Cronograma Detallado - Módulo 1}

\begin{table}[H]
\centering
\tiny
\begin{tabular}{|c|c|p{2.8cm}|c|p{3.8cm}|c|}
\hline
\textbf{Semana} & \textbf{Sesión} & \textbf{Actividad en Clase} & \textbf{Tiempo} & \textbf{Tarea Extraclase} & \textbf{Tiempo} \\
\hline

\multirow{4}{*}{\textbf{SEM. 1}} & 
\multirow{2}{*}{\textbf{S. 1}} & 
1.1 Bienvenida. Presentación del módulo y reto auténtico. & 0.5h &  &  \\
\cline{3-6}
& & 
1.2 Clase: Orientación al análisis de industria. Concepto de diagnóstico estratégico. & 1h &  &  \\
\cline{2-6}
& 
\multirow{2}{*}{\textbf{S. 2}} & 
1.3 Actividad: Presentación del caso/empresa. Sesión plenaria: ¿Qué sabemos? ¿Qué necesitamos saber? & 0.5h & 
1.4 Lectura preliminar de industria. Búsqueda de 3 reportes sectoriales. & 2h \\
\cline{3-6}
& & 
1.5 Clase: Fuentes de inteligencia competitiva. INEGI, reportes, análisis competitivo. & 1h &  &  \\
\hline

\multirow{4}{*}{\textbf{SEM. 2}} & 
\textbf{S. 3} & 
1.6 Taller: Búsqueda en bases de datos. Estudiantes localizan y descargan fuentes. Profesora guía selección. & 1.5h & 
1.7 Lectura crítica de 2 reportes. Subrayado de datos clave. & 2h \\
\cline{2-6}
& 
\multirow{2}{*}{\textbf{S. 4}} & 
1.8 Clase: Análisis de competencia. Mapeo de competidores. Propuestas de valor. & 1h &  &  \\
\cline{3-6}
& & 
1.9 Actividad: Construcción colaborativa de mapa de competidores. & 1h & 
1.10 Análisis individual de 1 competidor. 2 párrafos sobre posicionamiento. & 1.5h \\
\hline

\multirow{3}{*}{\textbf{SEM. 3}} & 
\textbf{S. 5} & 
1.11 Clase: FODA contextualizado. Evitar genéricos. Importancia de especificidad. & 1h & 
1.12 Borrador FODA individual usando información recopilada. & 2h \\
\cline{2-6}
& 
\textbf{S. 6} & 
1.13 Actividad: Equipos (3-4 personas). Socialización de borradores FODA. Refinamiento. & 1.5h &  &  \\
\cline{2-6}
& 
\textbf{S. 7} & 
1.14 Clase: Identificación de oportunidades estratégicas. De datos a insights. & 1h & 
1.15 Redacción de 3 oportunidades con justificación. & 1.5h \\
\hline

\textbf{SEM. 4} & 
\textbf{S. 8} & 
1.16 Reflexión crítica: Sesgos en nuestro análisis. ¿Qué datos priorizamos? ¿Quién invisibilizado? Discusión plenaria. & 1.5h &  &  \\
\hline

\textbf{SEM. 5} & 
\textbf{S. 9} & 
1.17 Entrega Reporte de Diagnóstico Situacional (20 puntos). Presentación oral (10 min/equipo). Retroalimentación de pares. & 2h &  &  \\
\hline

\end{tabular}
\end{table}

% ============================================
\section{Fases de la Secuencia Didáctica}
% ============================================

\subsection*{Tabla 4: Fase I - Actividades de Apertura (Sesiones 1-2)}

\begin{table}[H]
\centering
\small
\begin{tabular}{|p{2cm}|p{4.5cm}|p{4.5cm}|p{3.5cm}|}
\hline
\textbf{Sesión} & \textbf{Actividad} & \textbf{Objetivo Específico} & \textbf{Recursos} \\
\hline

\textbf{S. 1} & 
Presentación del módulo y reto auténtico & 
Situar el aprendizaje en contexto real. Activar disposición para el análisis. & 
Presentación visual del caso \\
\hline

\textbf{S. 1} & 
Clase expositiva: Orientación al análisis de industria & 
Establecer conceptos clave: diagnóstico, estrategia, contexto industrial. & 
Diapositivas, pizarrón \\
\hline

\textbf{S. 2} & 
Sesión plenaria: Presentación del caso. Preguntas orientadoras & 
Activar conocimientos previos. Identificar vacíos de información. & 
Caso impreso o digital \\
\hline

\textbf{S. 2} & 
Clase expositiva: Fuentes de inteligencia competitiva & 
Familiarizar con bases de datos, reportes sectoriales, plataformas de análisis. & 
Demostraciones en línea, listado de URLs \\
\hline

\textbf{Tarea} & 
Lectura preliminar y búsqueda de reportes sectoriales & 
Preparar el terreno para análisis profundo. Familiarización con lenguaje técnico. & 
Reportes en línea (INEGI, Euromonitor) \\
\hline

\end{tabular}
\end{table}

\subsection*{Tabla 5: Fase II - Actividades de Desarrollo (Sesiones 3-7)}

\begin{table}[H]
\centering
\small
\begin{tabular}{|p{2cm}|p{4.5cm}|p{4.5cm}|p{3.5cm}|}
\hline
\textbf{Sesión} & \textbf{Actividad} & \textbf{Objetivo Específico} & \textbf{Recursos} \\
\hline

\textbf{S. 3} & 
Taller práctico: Búsqueda en bases de datos & 
Desarrollar habilidad de localización y selección de información confiable. & 
Laboratorio con internet, tutoriales de acceso \\
\hline

\textbf{S. 3} & 
Tarea: Lectura crítica de reportes & 
Preparar análisis más profundo. Desarrollo de pensamiento crítico. & 
Reportes asignados \\
\hline

\textbf{S. 4} & 
Clase expositiva: Análisis de competencia & 
Enseñanza de métodos de mapeo competitivo y diferenciación de propuestas. & 
Ejemplos de mapas competitivos, casos reales \\
\hline

\textbf{S. 4} & 
Actividad colaborativa: Construcción de mapa de competidores & 
Aplicación inmediata de conceptos. Trabajo en equipo. Construcción colectiva. & 
Materiales para mapeo (post-its, papelógrafo o digital) \\
\hline

\textbf{S. 5} & 
Clase expositiva: FODA contextualizado & 
Evitar generalidades. Mostrar ejemplos de FODA genérico vs. especificado. & 
Ejemplos de análisis FODA (buenos y malos) \\
\hline

\textbf{S. 5} & 
Tarea: Borrador FODA individual & 
Práctica independiente. Análisis reflexivo de información recopilada. & 
Template de FODA, notas de sesiones previas \\
\hline

\textbf{S. 6} & 
Actividad: Refinamiento colaborativo en equipos & 
Mejora mediante retroalimentación de pares. Consolidación de análisis. & 
Borradores impresos o digitales \\
\hline

\textbf{S. 7} & 
Clase expositiva: Identificación de oportunidades & 
Enseñanza de cómo transformar datos en insights estratégicos. & 
Casos de empresas que identificaron oportunidades \\
\hline

\textbf{S. 7} & 
Tarea: Redacción de oportunidades estratégicas & 
Síntesis reflexiva. Fundamentación en datos. & 
Matriz de análisis, notas de sesiones \\
\hline

\end{tabular}
\end{table}

\subsection*{Tabla 6: Fase III - Actividades de Cierre (Sesiones 8-9)}

\begin{table}[H]
\centering
\small
\begin{tabular}{|p{2cm}|p{4.5cm}|p{4.5cm}|p{3.5cm}|}
\hline
\textbf{Sesión} & \textbf{Actividad} & \textbf{Objetivo Específico} & \textbf{Recursos} \\
\hline

\textbf{S. 8} & 
Reflexión crítica sobre sesgos en nuestro análisis & 
Desarrollar conciencia metacognitiva. Cuestionar datos priorizados. Reconocer invisibilizados. & 
Preguntas reflexivas en pizarrón, discusión facilitada \\
\hline

\textbf{S. 8} & 
Discusión plenaria sobre implicaciones & 
Integración de perspectiva socio-crítica. Reflexión colectiva. & 
Notas de análisis previo, facilitación de diálogo \\
\hline

\textbf{S. 9} & 
Entrega Reporte de Diagnóstico Situacional & 
Integración de todos los pasos previos en evidencia sumativa. & 
Rúbrica de evaluación \\
\hline

\textbf{S. 9} & 
Presentación oral en clase (10 min/equipo) & 
Comunicación efectiva de hallazgos. Demostración de aprendizaje. & 
Presentaciones en PowerPoint o similar \\
\hline

\textbf{S. 9} & 
Retroalimentación de pares & 
Aprendizaje desde perspectivas diferentes. Evaluación colaborativa. & 
Formato de retroalimentación (rúbrica simplificada) \\
\hline

\end{tabular}
\end{table}

% ============================================
\section{Evaluación y Evidencias de Aprendizaje}
% ============================================

\subsection*{Evaluación Formativa}

Durante cada sesión, la profesora observa y proporciona retroalimentación sobre:
\begin{itemize}
    \item Calidad de las preguntas que formulan los estudiantes
    \item Profundidad del análisis (¿van más allá de lo obvio?)
    \item Capacidad de sintetizar información de múltiples fuentes
    \item Disposición para reconocer limitaciones y sesgos
\end{itemize}

\subsection*{Evaluación Sumativa}

\textbf{Objeto de Aprendizaje 1:} Reporte de Diagnóstico Situacional (20 puntos)

\textbf{Componentes de la evaluación:}
\begin{itemize}
    \item \textbf{Análisis de tendencias de industria (5 puntos):} Identificación de dinámicas macroeconómicas y de mercado relevantes, con datos específicos.
    
    \item \textbf{Mapeo de competencia (5 puntos):} Claridad en identificación de competidores, diferenciación de propuestas de valor, competencia directa versus indirecta.
    
    \item \textbf{Análisis FODA contextualizado (5 puntos):} Especificidad, no generalidades. Conexión clara entre datos recopilados y conclusiones.
    
    \item \textbf{Reflexión crítica (5 puntos):} Explicitación de qué datos fueron priorizados, cuáles invisibilizados, y por qué. Reconocimiento de limitaciones del análisis.
\end{itemize}

% ============================================
\section{Recursos Requeridos}
% ============================================

\subsection*{Recursos Tecnológicos}
\begin{itemize}
    \item Acceso a bases de datos: INEGI (www.inegi.org.mx), Euromonitor, reportes sectoriales disponibles
    \item Herramientas de análisis: Excel, Google Sheets para organización de datos
    \item Plataforma de colaboración: Google Drive o Microsoft Teams para construcción colaborativa
\end{itemize}

\subsection*{Recursos Bibliográficos}
\begin{itemize}
    \item Reportes sectoriales específicos a la industria analizada (proporcionados o identificados por estudiantes)
    \item Lamb, Ch., Hair, J., y McDaniel, C. (2019). \textit{MKTG12}. Capítulos sobre análisis ambiental y competitivo.
\end{itemize}

\subsection*{Recursos Conceptuales}
\begin{itemize}
    \item Marco de inteligencia competitiva (fuentes de datos, indicadores clave)
    \item Matriz FODA y método de construcción
    \item Mapa de posicionamiento competitivo
\end{itemize}

% ============================================
\section{Diagrama de Conexión de Objetivos}
% ============================================

A continuación se presenta el diagrama que visualiza cómo se conectan y progresan los objetivos de aprendizaje a lo largo del Módulo 1:

\vspace{1cm}

\textit{}
\begin{figure}
    \centering
    \includegraphics[width=0.5\linewidth]{mermaid-diagram-2025-10-16-233921.png}
\end{figure}
\vspace{0.5cm}

\textit{El diagrama muestra el flujo desde requisitos previos iniciales (RP 1, RP 2) hacia pasos de procesamiento de información (PPI 1-5), mostrando cómo cada paso habilita el siguiente, culminando en la evidencia final: el Reporte de Diagnóstico Situacional que integra todos los PPI y culmina en el Objetivo Terminal de Aprendizaje.}

% ============================================
\section{Observaciones Finales}
% ============================================

Este diseño del Módulo 1 establece la base para los módulos posteriores. Al finalizar, los estudiantes habrán desarrollado:

\begin{enumerate}
    \item Competencia técnica en recopilación y análisis de información secundaria
    \item Disposición reflexiva para cuestionar datos y análisis convencionales
    \item Capacidad de síntesis que conecta múltiples datos en conclusiones estratégicas
    \item Preparación para el Módulo 2, donde profundizarán en comprensión del consumidor
\end{enumerate}
% =======================

% =======================
\begin{thebibliography}{9}

\bibitem[Apple(1979)]{Apple1979}
Apple, M. W. (1979). \textit{Ideology and curriculum}. Routledge \& Kegan Paul.

\bibitem[Cope \& Kalantzis(2012)]{Cope2012}
Cope, B., \& Kalantzis, M. (2012). \textit{New learning: Elements of a science of education} (2nd ed.). Cambridge University Press. \url{https://doi.org/10.1017/CBO9781139248532}

\bibitem[Gimeno Sacristán(2010)]{Gimeno2010}
Gimeno Sacristán, J. (2010). ¿Qué significa el currículum? \textit{Sinéctica}, (34), 11-43. \url{https://www.scielo.org.mx/scielo.php?script=sci_arttext&pid=S1665-109X2010000100009}

\bibitem[Marsh \& Willis(2007)]{Marsh2007}
Marsh, C. J., \& Willis, G. (2007). \textit{Curriculum: Alternative approaches, ongoing issues} (4th ed.). Pearson.

\bibitem[Siemens(2005)]{Siemens2005}
Siemens, G. (2005). Connectivism: A learning theory for the digital age. \textit{International Journal of Instructional Technology and Distance Learning}, \textit{2}(1), 3-10.

\end{thebibliography}

\end{document}