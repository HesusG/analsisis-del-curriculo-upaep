% ============================================
% PROYECTO FINAL: ANÁLISIS CRÍTICO DEL CURRÍCULUM
% Autor: Hesus Garcia Cobos
% Curso: Análisis del Currículum - UPAEP
% ============================================
\documentclass[12pt, letterpaper]{article}

% --- Idioma y codificación ---
% Nota: Instalar texlive-lang-spanish para babel completo
% Por ahora usamos configuración básica
\usepackage[utf8]{inputenc}
\usepackage[T1]{fontenc}

% --- Fuente Times New Roman ---
\usepackage{mathptmx}

% --- Bibliografía ---
\usepackage{natbib}
\bibliographystyle{apalike}

% --- Tablas y figuras ---
\usepackage{booktabs}
\usepackage{graphicx}
\usepackage{float}

% --- URLs y enlaces ---
\usepackage{url}
\usepackage[breaklinks=true]{hyperref}
\usepackage{csquotes}

% --- Formato de página (APA 7) ---
\usepackage{geometry}
\geometry{margin=2.5cm, headheight=15pt}  % Márgenes de 2.5cm según instrucciones

\usepackage{setspace}
\onehalfspacing  % Interlineado 1.5 según instrucciones

% --- Encabezados ---
\usepackage{fancyhdr}
\pagestyle{fancy}
\fancyhf{}
\fancyhead[R]{\thepage}
\renewcommand{\headrulewidth}{0pt}

% --- Justificación ---
\usepackage{ragged2e}

% --- Microtype para mejor tipografía ---
\usepackage{microtype}

% ============================================
% DATOS DEL TRABAJO
% ============================================
\newcommand{\myTitle}{Análisis y Propuesta de Adecuación Curricular: Del Aprendizaje Fragmentado al Diseño Integrado de Significado}
\newcommand{\myAuthor}{Hesus Garcia Cobos}
\newcommand{\myAffiliation}{Universidad Popular Autónoma del Estado de Puebla}
\newcommand{\myCourse}{147-139-PED605NE-702: Análisis del Currículum}
\newcommand{\myProfessor}{Dra. Melissa Isaaly Mendoza Bernabe}
\newcommand{\myDate}{29 de noviembre de 2025}

% ============================================
\begin{document}

% --- Página de título estilo APA ---
\thispagestyle{empty}
\vspace*{2in}
\begin{center}
    \textbf{\Large \myTitle} \\[24pt]
    \myAuthor \\[12pt]
    \myAffiliation \\[12pt]
    \myCourse \\[12pt]
    \myProfessor \\[12pt]
    \myDate
\end{center}

\newpage
\justifying

% ============================================
\section{Introducción}
% ============================================

El currículum, siguiendo a \citet{Gimeno2010}, es ``el contenido cultural que las instituciones educativas tratan de difundir en quienes las frecuentan, así como los efectos que dicho contenido provoque en sus receptores.'' Esta definición nos recuerda que el currículum no es un documento neutro ni estático: refleja decisiones sobre qué conocimiento se considera valioso y con qué fines se transmite.

El presente trabajo analiza el currículum del curso \textit{MT1001B: Descubrimientos del mercado para el desarrollo de estrategias}, un módulo de 60 horas impartido en el segundo semestre de la Entrada de Negocios del Tecnológico de Monterrey. A través de este análisis, propongo una adecuación curricular fundamentada en tres marcos teóricos complementarios:

\begin{enumerate}
    \item \textbf{Pedagogía de Multiliteracidades} \citep{Cope2009}: Los estudiantes no son receptores pasivos de información, sino diseñadores activos de significado a través de múltiples modos de comunicación (lingüístico, visual, espacial, auditivo, gestual).

    \item \textbf{Enfoque Socio-Crítico} \citep{Apple1979}: El currículum nunca es neutral; representa relaciones de poder e ideología. La pregunta central es: ``¿para qué enseñamos esto y a quién sirve?''

    \item \textbf{Conectivismo} \citep{Siemens2005}: En la era digital, el aprendizaje ocurre a través de redes y conexiones. Saber dónde encontrar información es tan importante como poseerla.
\end{enumerate}

Mi perspectiva integra estos tres elementos para proponer un currículum operacional que trascienda lo oficial y se convierta en práctica transformadora. El análisis distingue claramente entre el \textit{ser} (estado actual del curso) y el \textit{deber ser} (propuesta de mejora), siguiendo los niveles de concreción curricular que identifican \citet{Marsh2007}.

% ============================================
\section{Diagnóstico del Lugar e Institución}
% ============================================

\subsection{Contexto Institucional}

El curso MT1001B forma parte del modelo Tec21 del Tecnológico de Monterrey, un diseño curricular basado en competencias y retos. Se imparte a estudiantes de segundo semestre que inician su exploración en el área de negocios. El bloque tiene una duración de 5 semanas con 60 horas de contacto, divididas en tres componentes (véase Anexo \ref{anexo:estructura} para el detalle completo):

\begin{itemize}
    \item Módulo 1: Diagnóstico y desarrollo de la estrategia (20 horas)
    \item Módulo 2: Análisis del mercado para generación de insights (20 horas)
    \item Reto del bloque (20 horas)
\end{itemize}

\subsection{Perfil del Estudiante}

Los estudiantes que ingresan a este bloque presentan las siguientes características (véase Anexo \ref{anexo:competencias} para el detalle de competencias):

\begin{itemize}
    \item \textbf{Nivel de dominio}: Trabajan subcompetencias a nivel A (situaciones controladas, acompañamiento docente) y B (mayor autonomía, complejidad creciente).

    \item \textbf{Actitudes}: Buscan aplicación práctica inmediata; desean que lo aprendido tenga uso real en el mercado laboral.

    \item \textbf{Saberes previos}: Poseen capacidades de análisis y síntesis, interpretan gráficos, usan plataformas digitales y tienen habilidades básicas de investigación.

    \item \textbf{Tensión pedagógica}: Aunque requieren enfoque práctico, necesitan profundidad teórica reflexiva que conecte teoría y práctica.
\end{itemize}

\subsection{Diagnóstico Curricular: El ``Ser''}

Tras analizar la documentación curricular del curso (véanse Anexos \ref{anexo:secuencia} y \ref{anexo:evaluacion} para la secuencia didáctica y objetos de aprendizaje oficiales), identifico las siguientes características del estado actual:

\begin{table}[H]
\centering
\caption{Estado Actual del Currículum (El ``Ser'')}
\small
\begin{tabular}{p{3.5cm}p{10cm}}
\toprule
\textbf{Dimensión} & \textbf{Características Actuales} \\
\midrule
Estructura & Módulos independientes con secuencia lineal preestablecida \\
Contenido & Transmisión de conceptos teóricos mediante exposición docente \\
Actividades & Casos controlados, mini-casos simulados, ejercicios prediseñados \\
Evaluación & Exámenes conceptuales (Objeto 2), reportes individuales, rúbricas fijas \\
Tecnología & Herramientas como fin (Excel, PowerPoint) no como medio de creación \\
Rol del estudiante & Receptor de contenido, ejecutor de tareas predefinidas \\
Reflexión crítica & Mínima o ausente; énfasis en ``hacer bien'' no en ``cuestionar'' \\
\bottomrule
\end{tabular}
\end{table}

Esta estructura refleja lo que \citet{Cope2012} denominan ``aprendizaje antiguo'': conocimiento fragmentado, descontextualizado de las prácticas profesionales reales. Siguiendo a \citet{Apple1979}, me pregunto: ¿qué tipo de conocimiento se está legitimando con esta estructura tradicional? ¿A quiénes beneficia mantener la separación entre teoría académica y práctica del mercado?

\subsection{Problema Curricular Identificado}

El problema central es la \textbf{fragmentación entre teoría y práctica}, manifestada en:

\begin{enumerate}
    \item Desconexión entre los contenidos conceptuales y los problemas reales del mercado
    \item Ausencia de oportunidades para que los estudiantes diseñen significado multimodal
    \item Falta de reflexión crítica sobre el poder e ideología en las decisiones de marketing
    \item Evaluación centrada en reproducción de contenidos, no en creación de conocimiento
\end{enumerate}

% ============================================
\section{Objetivo del Proyecto}
% ============================================

\subsection{Objetivo General}

Transformar el Módulo 1 del curso MT1001B de una estructura tradicional y fragmentada hacia un modelo integrado y auténtico donde los estudiantes se conviertan en diseñadores activos de significado, conecten múltiples fuentes de información, y reflexionen críticamente sobre las implicaciones ideológicas de sus análisis de mercado.

\subsection{Objetivos Específicos}

\begin{enumerate}
    \item Rediseñar las actividades del módulo para integrar los principios de multiliteracidades.
    \item Incorporar momentos de reflexión crítica basados en el enfoque socio-crítico de Apple.
    \item Aplicar principios conectivistas para que los estudiantes construyan redes de conocimiento.
    \item Proponer un sistema de evaluación que valore la creación de significado, no solo la reproducción.
\end{enumerate}

% ============================================
\section{Diseño de la Propuesta Curricular}
% ============================================

\subsection{Nivel de Concreción}

Esta propuesta opera en el \textbf{nivel micro (áulico)} de concreción curricular. No modifica el plan de estudios institucional (nivel macro) ni el diseño programático del bloque (nivel meso), sino que propone adecuaciones dentro del aula que el docente puede implementar manteniendo los objetivos de aprendizaje oficiales.

\subsection{Comparación: Ser vs Deber Ser}

\begin{table}[H]
\centering
\caption{Transformación Curricular: Del Ser al Deber Ser}
\small
\begin{tabular}{p{2.5cm}p{5.5cm}p{5.5cm}}
\toprule
\textbf{Aspecto} & \textbf{Ser (Estado Actual)} & \textbf{Deber Ser (Propuesta)} \\
\midrule
Estructura & Módulos independientes & Secuencia integrada donde cada fase construye sobre la anterior \\
Epistemología & Conocimiento transmitido y fragmentado & Conocimiento multimodal, colaborativo, construido \\
Casos & Mini-casos ficticios controlados & Problemas auténticos con empresas reales o emprendimientos locales \\
Rol estudiante & Receptor pasivo & Diseñador activo de significado \\
Reflexión & Ausente & Integrada: ``¿A quién sirve este análisis?'' \\
Evaluación & Exámenes y reportes tradicionales & Portafolios multimodales, presentaciones, productos reales \\
Tecnología & Herramientas para entregar tareas & Medio para crear, conectar y comunicar \\
\bottomrule
\end{tabular}
\end{table}

\subsection{Integración de Marcos Teóricos en la Propuesta}

Cada actividad del módulo rediseñado integra los tres marcos teóricos:

\begin{table}[H]
\centering
\caption{Integración de Marcos Teóricos en Actividades}
\small
\begin{tabular}{p{3cm}p{3.5cm}p{3.5cm}p{3.5cm}}
\toprule
\textbf{Actividad} & \textbf{Multiliteracidades} & \textbf{Socio-Crítico} & \textbf{Conectivismo} \\
\midrule
Diagnóstico de industria & Análisis visual de datos, infografías & ``¿Qué datos priorizamos y cuáles invisibilizamos?'' & Conexión con múltiples fuentes (INEGI, reportes, redes) \\
Mapeo de competidores & Diseño de mapas visuales interactivos & ``¿A quién beneficia esta segmentación?'' & Colaboración en plataformas digitales \\
Análisis FODA & Presentación multimodal (video, visual, escrito) & ``¿Qué voces están ausentes en este análisis?'' & Retroalimentación en red de pares \\
Reporte final & Portafolio digital integrado & Reflexión sobre poder e ideología & Publicación en comunidad de práctica \\
\bottomrule
\end{tabular}
\end{table}

\subsection{Secuencia Didáctica Propuesta}

La secuencia se organiza en cuatro fases que corresponden a los procesos de conocimiento de \citet{Cope2015}: Experimentar, Conceptualizar, Analizar y Aplicar.

\begin{table}[H]
\centering
\caption{Secuencia Didáctica del Módulo 1 Rediseñado}
\small
\begin{tabular}{p{2cm}p{3cm}p{8.5cm}}
\toprule
\textbf{Semana} & \textbf{Fase} & \textbf{Actividades Principales} \\
\midrule
1 & \textbf{Experimentar} & Inmersión en reto auténtico. Investigación de industria usando múltiples fuentes. Creación de muro colaborativo digital. \\
2 & \textbf{Conceptualizar} & Análisis de competencia. Construcción de mapas visuales. Discusión crítica: ``¿Qué no nos dicen estos datos?'' \\
3 & \textbf{Analizar} & FODA contextualizado. Reflexión sobre sesgos. Identificación de oportunidades estratégicas. \\
4-5 & \textbf{Aplicar} & Elaboración de portafolio multimodal. Presentación ante pares. Reflexión crítica final. \\
\bottomrule
\end{tabular}
\end{table}

\subsection{Impacto Esperado}

La implementación de esta adecuación curricular tendría los siguientes impactos:

\begin{enumerate}
    \item \textbf{En los estudiantes}: Mayor engagement al trabajar con problemas auténticos; desarrollo de competencias de pensamiento crítico; capacidad de crear y comunicar en múltiples modos.

    \item \textbf{En el aprendizaje}: Conocimiento más profundo y transferible; conexiones significativas entre teoría y práctica; reflexión sobre las implicaciones éticas del marketing.

    \item \textbf{En la práctica docente}: Rol facilitador más que transmisor; evaluación más rica y diversificada; oportunidad de innovación pedagógica dentro del marco institucional.
\end{enumerate}

% ============================================
\section{Reflexión Final}
% ============================================

Este proceso de análisis y diseño curricular me ha permitido comprender de manera profunda los niveles de concreción curricular y su impacto en la práctica educativa cotidiana.

\subsection{Sobre los Niveles de Concreción Curricular}

\citet{Marsh2007} distinguen entre el currículum macro (políticas nacionales), meso (diseño institucional) y micro (implementación en aula). Mi propuesta opera en el nivel micro, lo cual tiene implicaciones importantes:

\begin{itemize}
    \item \textbf{Posibilidad}: Un docente puede implementar cambios significativos sin modificar el programa oficial.
    \item \textbf{Limitación}: Los cambios profundos requieren articulación con los niveles meso y macro.
    \item \textbf{Tensión}: El currículum operacional (lo que realmente sucede en el aula) puede diferir del oficial.
\end{itemize}

Reconozco que esta propuesta, al ser de nivel micro, enfrenta restricciones estructurales. Sin embargo, como señala \citet{Stenhouse1975}, el currículum es hipótesis a probar en la práctica, y el docente es el investigador de su propia práctica.

\subsection{Sobre el Proceso de Análisis}

Realizar este trabajo me permitió:

\begin{enumerate}
    \item \textbf{Integrar teoría y práctica}: Aplicar los marcos de Cope, Kalantzis, Apple y Siemens no como contenido a memorizar, sino como lentes para analizar y transformar una realidad curricular concreta.

    \item \textbf{Cuestionar lo naturalizado}: Gracias al enfoque socio-crítico, pude preguntarme por qué el currículum actual tiene la forma que tiene y a quiénes beneficia o perjudica.

    \item \textbf{Valorar la complejidad}: El diseño curricular no es un proceso técnico neutral; involucra decisiones epistemológicas, políticas y éticas.
\end{enumerate}

\subsection{Implicaciones para mi Desarrollo Profesional}

Como futuro profesional de la educación, este análisis me deja aprendizajes fundamentales:

\begin{itemize}
    \item El currículum es un espacio de disputa y posibilidad, no un documento fijo.
    \item La reflexión crítica constante es esencial para no reproducir desigualdades.
    \item Los estudiantes merecen ser tratados como diseñadores de significado, no como receptáculos vacíos.
    \item La tecnología debe ser medio de creación, no fin en sí misma.
\end{itemize}

Finalmente, me pregunto: ¿qué pasaría si más docentes se permitieran cuestionar el currículum que implementan? Quizás, como sugiere \citet{Apple1979}, podríamos construir una educación que no solo transmita cultura, sino que empodere a los estudiantes para transformarla.

% ============================================
% REFERENCIAS
% ============================================
\newpage
\bibliography{referencias}

% ============================================
% ANEXOS
% ============================================
\newpage
\appendix
\section*{Anexos}
\addcontentsline{toc}{section}{Anexos}

Los siguientes anexos presentan la documentación oficial del curso MT1001B del Tecnológico de Monterrey, que constituye el estado actual (``ser'') del currículum analizado en este trabajo.

% --------------------------------------------
\newpage
\section{Estructura y Organización del Curso}
\label{anexo:estructura}

\subsection*{Datos Generales del Bloque}

\begin{table}[H]
\centering
\caption{Ficha Técnica del Bloque MT1001B}
\begin{tabular}{p{4cm}p{9cm}}
\toprule
\textbf{Campo} & \textbf{Descripción} \\
\midrule
Clave & MT1001B \\
Nombre & Descubrimientos del mercado para el desarrollo de estrategias \\
Ubicación curricular & Semestre 2, Entrada de Negocios \\
Duración total & 5 semanas \\
Horas de contacto & 60 horas \\
Modelo educativo & Tec21 (basado en competencias y retos) \\
\bottomrule
\end{tabular}
\end{table}

\subsection*{Distribución de Componentes}

El bloque se organiza en tres componentes de igual duración:

\begin{table}[H]
\centering
\caption{Componentes del Bloque MT1001B}
\begin{tabular}{clc}
\toprule
\textbf{Componente} & \textbf{Nombre} & \textbf{Horas} \\
\midrule
Módulo 1 & Diagnóstico y desarrollo de la estrategia & 20 \\
Módulo 2 & Análisis del mercado para la generación de insights & 20 \\
Reto & Reto del bloque & 20 \\
\midrule
& \textbf{Total} & \textbf{60} \\
\bottomrule
\end{tabular}
\end{table}

\subsection*{Cronograma del Módulo 1}

\begin{table}[H]
\centering
\caption{Distribución de Sesiones del Módulo 1}
\small
\begin{tabular}{ccp{8cm}c}
\toprule
\textbf{Semana} & \textbf{Sesión} & \textbf{Contenido Principal} & \textbf{Horas} \\
\midrule
1 & 1 & Bienvenida. Orientación al consumidor. Presentación del caso. & 2 \\
1 & 2 & Identificación de necesidades. Análisis de situación. & 2 \\
1 & 3 & Diferenciación y propuesta de valor. Objetivos de mercadotecnia. & 2 \\
1 & 4 & Estrategias del mercado meta. Segmentación. & 2 \\
2 & 5 & Propuestas de solución. Objeto de aprendizaje 1 (mini caso). & 2 \\
3 & 6 & Objeto de aprendizaje 2: Evaluación conceptual. & 2 \\
3 & 7 & Estructura del plan de mercadotecnia. Mezcla de mercadotecnia. & 3 \\
3 & 8 & Productos y servicios. Empaque y etiqueta. & 2 \\
4 & 9 & Plaza, distribución multicanal y omnicanal. Retailing. & 2 \\
4 & 10 & Lógica de servicio. Taller de design thinking. & 3 \\
\midrule
& & \textbf{Total Módulo 1} & \textbf{20} \\
\bottomrule
\end{tabular}
\end{table}

% --------------------------------------------
\newpage
\section{Competencias y Subcompetencias}
\label{anexo:competencias}

El bloque MT1001B desarrolla tres competencias principales del Área de Negocios, cada una con subcompetencias específicas trabajadas a niveles de dominio A y B.

\subsection*{Competencia 1: Inteligencia de Negocios (SNEG0400)}

\textit{Definición}: Analizar de forma crítica información cuantitativa y cualitativa para la toma de decisiones, utilizando razonamiento matemático, técnicas de programación, métodos estadísticos y herramientas tecnológicas de vanguardia.

\begin{table}[H]
\centering
\caption{Subcompetencia SNEG0405: Representación y Visualización}
\small
\begin{tabular}{cp{12cm}}
\toprule
\textbf{Nivel} & \textbf{Descripción del Dominio} \\
\midrule
A & El alumno aplica técnicas de representación para elaborar reportes gráficos digitales que faciliten la toma de decisiones, utilizando herramientas computacionales y técnicas visuales. \\
B & El alumno realiza presentaciones visuales efectivas, elabora reportes dinámicos y gráficos digitales para la toma de decisiones con casos de negocio simulados. \\
\bottomrule
\end{tabular}
\end{table}

\subsection*{Competencia 2: Mercados y Oportunidades de Negocio (SNEG0700)}

\textit{Definición}: Crear oportunidades de negocio a través de la identificación y satisfacción de las necesidades del mercado.

\begin{table}[H]
\centering
\caption{Subcompetencias del Módulo 1 y 2}
\small
\begin{tabular}{clp{9cm}}
\toprule
\textbf{Código} & \textbf{Nombre} & \textbf{Nivel A} \\
\midrule
SNEG0703 & Plan de Mercadotecnia & Diseñar un plan de mercadotecnia básico que incluya análisis de situación, FODA, objetivos y estrategias. \\
SNEG0701 & Inteligencia de Mercados & Identificar necesidades del mercado mediante obtención y análisis de información primaria y secundaria. \\
\bottomrule
\end{tabular}
\end{table}

\subsection*{Competencia 3: Emprendimiento Innovador (SEG0200)}

\textit{Definición}: Generar soluciones innovadoras y versátiles en entornos cambiantes que crean valor e impactan positivamente a la sociedad.

\begin{table}[H]
\centering
\caption{Subcompetencia SEG0201: Innovación}
\small
\begin{tabular}{cp{12cm}}
\toprule
\textbf{Nivel} & \textbf{Descripción del Dominio} \\
\midrule
A & Generar soluciones innovadoras ante problemáticas del entorno a través de un proceso cíclico que incorpora la validación y el aprendizaje. El alumno actúa proactivamente mostrando empatía con las necesidades del usuario. \\
\bottomrule
\end{tabular}
\end{table}

% --------------------------------------------
\newpage
\section{Secuencia Didáctica Oficial}
\label{anexo:secuencia}

El Módulo 1 sigue un modelo pedagógico de 6 fases con retroalimentación continua, basado en el aprendizaje experiencial.

\subsection*{Modelo Pedagógico del Módulo}

\begin{table}[H]
\centering
\caption{Fases del Modelo Didáctico (Modelo 5E Adaptado)}
\begin{tabular}{clp{8cm}}
\toprule
\textbf{Fase} & \textbf{Nombre} & \textbf{Descripción} \\
\midrule
1 & Empatiza & El profesor presenta el caso de estudio. El estudiante identifica la información y establece la situación. \\
2 & Explica & El experto explica contenidos conceptuales y sienta bases para contenidos procedimentales y actitudinales. \\
3 & Explora & Los estudiantes profundizan en el análisis aplicando herramientas vistas en clase. \\
4 & Elabora & Los estudiantes elaboran reportes de hallazgos: ejecutivos, gráficos, dinámicos y digitales. \\
5 & Innova & Sesión plenaria para generación de propuestas innovadoras mediante design thinking. \\
6 & Evalúa & Evaluación de aprendizajes con retroalimentación continua. \\
\bottomrule
\end{tabular}
\end{table}

\subsection*{Contenidos por Tipo}

\begin{itemize}
    \item \textbf{Conceptuales}: Orientación al consumidor, identificación de necesidades, diferenciación, propuesta de valor, estructura del plan de mercadotecnia, mezcla de mercadotecnia (4 P's).
    \item \textbf{Procedimentales}: Elaboración de reportes ejecutivos, gráficos dinámicos y digitales; identificación de oportunidades a partir de problemas detectados.
    \item \textbf{Actitudinales}: Actitud proactiva ante la generación de soluciones para la problemática presentada.
\end{itemize}

% --------------------------------------------
\newpage
\section{Objetos de Aprendizaje y Evaluación}
\label{anexo:evaluacion}

El Módulo 1 incluye cinco objetos de aprendizaje formales que estructuran la evaluación del estudiante.

\begin{table}[H]
\centering
\caption{Objetos de Aprendizaje del Módulo 1}
\small
\begin{tabular}{clcp{6cm}}
\toprule
\textbf{OA} & \textbf{Nombre} & \textbf{Pts} & \textbf{Descripción} \\
\midrule
1 & Diagnóstico básico de mercadotecnia & 5 & Análisis de mini caso: identificar industria, competencia, propuesta de valor y síntomas de la empresa. \\
2 & Evaluación conceptual & 5 & Examen sobre: Overview of Marketing, Strategic Planning, Segmenting and Targeting Markets. \\
3 & Ejes rectores de mercadotecnia & 2 & Matriz de 4 cuadrantes con las 4 P's aplicadas a un mini caso. \\
4 & Guía del plan de mercadotecnia & 2 & Investigación secundaria sobre estructura del plan de mercadotecnia. \\
5 & Taller de design thinking & 6 & Desarrollo de pensamiento creativo e innovador mediante metodología design thinking. \\
\midrule
& \textbf{Total Módulo 1} & \textbf{20} & \\
\bottomrule
\end{tabular}
\end{table}

\subsection*{Perfil de Ingreso del Estudiante}

Los estudiantes que ingresan al bloque MT1001B presentan las siguientes características:

\begin{itemize}
    \item \textbf{Ubicación curricular}: Semestre 2 de la Entrada de Negocios.
    \item \textbf{Trayectoria}: Etapa temprana de exploración en negocios.
    \item \textbf{Nivel de dominio}: Listos para trabajar subcompetencias a Nivel A (situaciones controladas, con acompañamiento docente) y Nivel B (mayor autonomía, complejidad creciente).
    \item \textbf{Autonomía}: Requieren guía para integrar conocimientos básicos que permitan formular un plan básico de mercadotecnia.
\end{itemize}

\subsection*{Bibliografía Oficial del Curso}

El curso utiliza como referencia principal:

\begin{quote}
Lamb, C. W., Hair, J. F., \& McDaniel, C. (2019). \textit{MKTG12} (12th ed.). Cengage Learning.
\end{quote}

Capítulos específicos evaluados:
\begin{itemize}
    \item An Overview of Marketing
    \item Strategic Planning for Competitive Advantage
    \item Segmenting and Targeting Markets
    \item Product Concepts
\end{itemize}

\end{document}
